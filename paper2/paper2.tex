\documentclass{article}

\title{An agent-based model of military mechanization}

\usepackage[citestyle=authoryear, style=authoryear]{biblatex}
\addbibresource{paper2.bib}

\author{Tom Wallace}

\begin{document}

\maketitle

\section{Introduction}

This paper presents an agent-based model of military mechanization. First, it
provides a theoretical discussion of military mechanization, states the research
hypothesis, and explains why agent-based modeling is an appropriate research strategy.
Then, it details an original agent-based model; presents results; and discusses
their implications for the academic debate on military mechanization.

In summary, the orthodox 

\section{Background}

States differ in the composition of their militaries, and in particular, of
their ground forces.\footnote{States may have multiple organizations that
conduct ground combat operations: e.g., the United States operates both an Army
and a Marine Corps as independent services. Henceforth, this paper uses the
terms ``army'' and ``ground forces'' interchangeably.} One notable axis of comparison 
is \textit{mechanization}: the degree to which the state relies on infantry 
vs. ground combat vehicles. Treating mechanization as a continuous variable, 
one extreme is an army consisting exclusively of dismounted, small 
arms-wielding infantry units; the other extreme is an army consisting 
solely of tank units, armored personnel carriers, artillery, and the like.
Clearly, no modern state embodies either extreme, but the 1970s-era Vietcong are
a good example of a low-mechanization army and the 1980s-era Israeli army of
high mechanization.

The consequences of choices regarding mechanization are extreme. To quote former
U.S. Secretary of Defense Donald Rumsfeld, ``you go to war with the army you
have.''

The causes of mechanization 

\cite{sechser2010army}
\cite{biddle2004military}
\cite{gartzke2001democracy}
\cite{farrell2001transitional}

\section{Model}

\section{Results}

\section{Discussion}
\printbibliography[heading=bibnumbered]

\end{document}
