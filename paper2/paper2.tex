\documentclass{article}


\usepackage[citestyle=authoryear, style=authoryear]{biblatex}
\addbibresource{paper2.bib}

\title{An agent-based model of military mechanization}

\author{Tom Wallace}

\begin{document}

\maketitle

\section{Introduction}

This paper presents an agent-based model of military mechanization. First, it
provides a theoretical discussion of military mechanization, states the research
hypothesis, and explains why agent-based modeling is an appropriate research strategy.
Then, it details an original agent-based model; presents results; and discusses
their implications for the academic debate on military mechanization.

In summary, the orthodox view of military mechanization is that it is a
reflexive process: states choose their mechanization posture largely based on
that neighbors and enemies. This paper argues
that this view---based on traditional longitudinal data analysis---is ripe for
agentization: 

\section{Background}

\subsection{Military mechanization}

States differ in the composition of their military ground forces.\footnote{States may have multiple organizations that
conduct ground combat operations: e.g., the United States operates both an Army
and a Marine Corps as independent services. Henceforth, this paper uses the
terms ``army'' and ``ground forces'' interchangeably.} One notable axis of comparison 
is \textit{mechanization}: the degree to which an army consists of infantry 
vs. combat vehicles. Treating mechanization as a continuous variable, 
one extreme is an army consisting exclusively of dismounted, small 
arms-wielding infantry units; the other extreme is an army consisting 
solely of tank units, armored personnel carriers, artillery, and the like.
Clearly, no modern state embodies either extreme, but the 1970s-era Vietcong are
a good example of a low-mechanization army and the 1980s-era Israeli army of
high mechanization.

The consequences of choices regarding mechanization are large. There is
academic consensus that defense policy---including but not limited to
mechanization---conditions battlefield effectiveness. 
\cite{lyall2009rage} finds that highly mechanized militaries are less effective
at counterinsurgency, while \cite{biddle2004military} argues that
mechanization-enabled mobility aids conventional warfighting. It thus
pays to have an army that is designed to win the
kinds of
conflicts that a state is likely to face. Lacking a crystal ball, however,
leaders cannot always forecast conflict, and armies are large, complex
organizations that typically are held to change only infrequently and slowly at
that, even if there is a pressing need (\cite{murray1998military},
\cite{locher2004victory}, \cite{zegart2000flawed}). A state suddenly faced with
a war for which its mechanization posture is ill-suited can either concede or
pay a cost in blood and treasure; what it cannot do is
quickly overcome the constraints imposed by past mechanization decisions. 
A quote from then-Secretary of Defense Donald Rumsfeld---made in 2004, when the U.S. military
was confronted in Iraq with the consequences of past choices regarding
mechanization---emphasizes the point:
``you go to war with the army you have'' (\cite{schmitt_2004}). 

The causes of military mechanization are disputed. \textit{Realists} believe 
that mechanization choices are driven by security environment. States
choose mechanization policies that they believe will allow them to prevail against potential adversaries, or to send a
signal of deterrence, or to learn from the perceived mistakes of the last war
(\cite{mearsheimer1983conventional};
\cite{huth1988extended}; \cite{murray2011military}). The other school of thought is more
diverse and so harder to name, but may be termed 
\textit{institutionalists}. They believe that factors other than 
strategic calculation determine military mechanization. Domestic political
institutions---e.g., democracy vs. autocracy, politically stable vs.
coup-prone, the state of civil-military relations, and so on---may influence 
force structure (\cite{reiter2002democracies}; \cite{quinlivan1999coup};
\cite{talmadge2015dictator}; \cite{brooks2008shaping}). Economic factor
endowments may play a role: e.g., capital-rich states may mechanize more
than labor-rich states (\cite{gartzke2001democracy}). Lastly,
ideology (\cite{van1984cult}) and culture (e.g., \cite{pollack2004arabs}) are
hard-to-measure but perhaps influential. Consider Ireland: between the 1920s and
1940s, it expended a large portion of its defense budget on building
a small number of tanks. They were wholly insufficient to repel a British 
invasion (their ostensible purpose), and
Ireland would have been better-served to pursue a low-mechanization guerrilla
defense strategy, but tanks were seen as a prestigious signifier of a ``real''
professional military, an important factor for the young Irish state
(\cite{farrell1998professionalization};
\cite{farrell2001transnational}).

\cite{sechser2010army} is the landmark study on military mechanization. They
assemble a longitudinal dataset at the country-year level of the mechanization
of all states between 1979 and 2001, and conduct regression analysis to identify
the effect of covariates on mechanization. They conclude that ``choices about
mechanization are strongly associated with a state's security environment.''
The more mechanized a state's geographic neighbors and enemies, the higher the state's 
own mechanization; also, states
learn from defeat in insurgency by subsequently decreasing their mechanization.

\subsection{The case for agent-based modeling}

The findings of \cite{sechser2010army} are ripe for agentization. The
conclusions drawn implicitly claim
that states are autonomous agents; that they have
cognition regarding how to perceive the outside world in terms of mechanization
and how to adjust their own mechanization to it; that spatiality and
neighborhood matter; that inter-agent relationships (e.g. whether you are my
enemy or not) matter; and that
agents learn from experience. Agent-based modeling is uniquely well-suited to
represent such phenomena (\cite{gilbert2005simulation};
\cite{miller2009complex}).

Importantly, if the findings of \cite{sechser2010army} are correct and states
largely determine their own mechanization levels on the basis of what other
states are doing, a positive feedback loop exists. State A has high
mechanization, causing enemy State B to raise their mechanization, causing State
A to raise their mechanization... and so on. This feedback loop---often called
the ``security dilemma''---has been extensively studied, including by Thomas Schelling, one of the pioneers
of agent-based modeling (\cite{schelling1960strategy};
\cite{schelling2006micromotives}).

\subsection{Research question and strategy}

I hypothesize that the arguments of \cite{sechser2010army} are incomplete and
that some omitted variable or set of variables acts as a ``brake'' on military
mechanization.\footnote{By ``omitted'', I mean that they conclude that the variable
is not very important for mechanization, not that they do not address the
variable at all.} An agent-based model is used to test this hypothesis. In broad
terms, agents (states) are defined to have the cognitive and learning behavior
implicitly claimed by \cite{sechser2010army}. The model is initialized with
real-world data from 1979 and allowed to run until 2001. The micro- and
macro-outcomes are then validated against real-world 2001 data. If
\cite{sechser2010army} are correct, the model's 2001 outcomes should match
real-world 2001 outcomes. If the model mechanization levels in 2001 are much
higher, then their claimed behavioral rules are incomplete and fail to include
some ``brake'' on mechanization, matching my hypothesis. Proving exactly what
that missing variable or set of variables is is outside the scope of this model
and paper, though it is discussed.

\section{Model}

This section of the paper describes the model. It does not use the ODD protocol
of \cite{grimm2006standard} but covers many of the same points.

\subsection{Agents}

There is one agent type: states. Agents exist at a fixed location and do not
move. The composition of agents changes throughout the 

\subsection{Environment}

Agents exist in a spatial environment. Spatiality is represented by neighbor
relationships rather than a raster or vector GIS map. Two states are defined as
being neighbors if they share a contiguous land border or a stretch of water
less than 400 miles. Every agent has an attribute \texttt{neighbor\_list}, a
list in which it stores the names of neighboring agents for that time-step. 

\subsection{Initialization and Time}

\subsection{Dynamics and Interactions}

\section{Results}

\section{Discussion}
\printbibliography[heading=bibnumbered]

\end{document}
