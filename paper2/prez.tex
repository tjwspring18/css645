\documentclass{beamer}
\usetheme{Madrid}
\usecolortheme{spruce}

\usepackage{graphicx}
\graphicspath{{./img/}}

\title{An agent-based model of military mechanization}
\subtitle{CSS 645}
\author{Tom Wallace}
\institute{George Mason University}
\date{Spring 2018}

\begin{document}

\frame{\titlepage}

\begin{frame}
	\frametitle{Military mechanization}

	States differ in the composition of their militaries, particularly ground forces \\~\\

	One axis of comparison is \textbf{mechanization}
	\begin{itemize}
		\item \small The degree to which an army is comprised of infantry vs. ground combat vehicles \\~\\
	\end{itemize}

	Historical examples help illustrate the general idea
	\begin{itemize}
		\item \small 1970s Vietcong (low mechanization) vs. 1980s Israel (high mechanization)
		\item \small 1980s U.S. (higher mechanization) vs. present-day U.S. (lower mechanization)
	\end{itemize}
\end{frame}

\begin{frame}
	\frametitle{The consequences of choices regarding force structure are great}
	Academic and professional consensus that choices regarding military
	composition (including but not limited to mechanization) condition
	battlefield effectiveness \\~\\

	Size, cost, and complexity dictate that states can only slowly alter 
	military composition, even under great duress \\~\\

	``You go to war with the army you have'' \\~\\
\end{frame}

\begin{frame}
	\frametitle{The determinants of mechanization are disputed}
	One view stresses \textbf{security environment}: states adjust
	mechanization to be able to prevail in conflict
	\begin{itemize}
		\item \small Neighbors' and enemys' mechanization (e.g. 1930s
			France and Germany)
		\item \small Terrain (e.g. Switzerland)
		\item \small Insurgency (e.g. U.S.)\\~\\
	\end{itemize}

	Another view stresses \textbf{political, economic, and cultural
	factors}: states are not pure security-maximizers
	\begin{itemize}
		\item \small Regime type (e.g. democracy vs. autocracy)
		\item \small Economic endowments (e.g. capital vs. labor)
		\item \small Cultural norms (e.g. Irish tanks)\\~\\
	\end{itemize}

	Sechser and Saunders 2010 conclude the former matters much more than the
	latter
	\begin{itemize}
		\item \small Developed state-year level dataset on
			mechanization, 1979-2001
		\item \small Traditional econometric analysis (OLS)
		\item \small Cultural norms (e.g. Irish tanks)\\~\\
	\end{itemize}

\end{frame}

\begin{frame}
	\frametitle{Viewing their findings through an ABM lens}

	Emphasis on \textbf{spatiality}: what are my neighbors doing? \\~\\

	My actions affect your actions affect my actions... $\to$ \textbf{positive feedback loops} \\~\\

	Estimated statistical coefficients implicitly claim particular cognitive model
	\begin{itemize}
		\item \small If neighbors mechanization goes up by $X$, I raise my mechanization by $\beta$ \\~\\
	\end{itemize}

	My research question: do their findings hold up under an ABM formulation?
\end{frame}

\begin{frame}
	\frametitle{Model description}
	Agent = state actor\\~\\

	Attributes:
	\begin{itemize}
		\item \small Mechanization level (vehicles : infantry ratio,
			IISS data)
		\item \small Geographic neighbors (contiguous border per COW dataset)
		\item \small Enemies (defined based on 10-year MID history)
		\item \small Terrain (Fearon and Laitin 2004) \\~\\
	\end{itemize}

	Cognition regarding mechanization
	\begin{itemize}
		\item \small Takes into account neighbors, enemies, terrain,
			wars
		\item \small Agents observe lagged global behavior (2 years prior) and calculate
			weighted average of neighbor mechanization; weighted
			average of enemy mechanization; own terrain; and wars
		\item \small For each of the above, adjust own mechanization by some
			coefficient
		\item \small Coefficients drawn from statistical distribution
			found by Sechser and Saunders, randomly assigned to
			agents (heterogeneity)
	\end{itemize}
\end{frame}

\begin{frame}
	\frametitle{Model description}
	Initialized in 1979 \\~\\

	Proceeds in two-year timesteps until 2001 \\~\\

	Every turn, agents perceive neighbors, enemies, terrain, and wars, and
	adjust own mechanization accordingly \\~\\

	Because agent cognition has random element, model is stochastic and must
	be run many times \\~\\

	General research strategy: initialize with values from 1979, run, see if model results in 2001 match reality at both
	actor- and system-level
\end{frame}

\begin{frame}
	\frametitle{Example}
\end{frame}

\begin{frame}
	\frametitle{Findings}

	Model results in significantly higher mechanization at both system- and
	agent-level than reality

	INSERT IMAGES
\end{frame}

\begin{frame}
	\frametitle{Influence of positive feedback loops}

	Results stem from positive feedback loop: you increase your
	mechanization, which causes me to increase my mechanization, which
	causes you to increase your mechanization... \\~\\

	Modeled system lacks countervailing negative feedback loop \\~\\

	Conclusion: some of the factors dismissed by Sechser and Saunders must
	play a role in mitigating rate-of-increase in the real-world system
	\\~\\

	Cannot say \textit{which} factors, but the system-level point remains

\end{frame}

\begin{frame}
	\frametitle{Verification, validation, and other topics}

	In our taxonomy: analyzed-analyzed model \\~\\

	Validation central to research strategy: comparison of model-2001 to
	reality-2001 \\~\\

	Significant amount of coding work was dealing with fact that composition
	of international system changed over time (e.g., 1991)

	\begin{itemize}
		\item \small Inheritance
	\end{itemize}
\end{frame}

\begin{frame}
	\frametitle{Possible extensions}

	Alliances \\~\\

	More sophisticated cognition \\~\\

	Incorporation of other force structure choices (e.g. human capital) \\~\\
\end{frame}

\end{document}
